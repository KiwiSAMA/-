\documentclass[12pt,a3paper]{article}
\usepackage[space]{ctex}
\usepackage{geometry}
\usepackage{mathtools}
\usepackage{tasks}
\settasks{
    counter-format={tsk[A].},
	label-offset={0.4em},
	label-align=left,
	column-sep={2pt},
    item-indent={1pt},
    before-skip={-0.7em},
    after-skip={-0.7em}
    }

\setlength{\parindent}{0pt}
\geometry{scale=0.8}
\title{第二学期期末数学试题(样)}
\date{}
\begin{document}
\maketitle



1、点(2,-1,3)关于坐标面$xoy$的对称点是\hfill(\qquad)
\begin{tasks}(4)
    \task $(2,1,3)$
    \task $(-2,-1,3)$
    \task $(2,-1,-3)$
    \task $(-2,1,-3)$
    \\
\end{tasks}
1、点(2,-1,3)关于平面y=1的对称点是\hfill(\qquad)
\begin{tasks}(4)
    \task $(2,3,3)$
    \task $(-2,-1,3)$
    \task $(2,-1,-3)$
    \task $(-2,1,-3)$
    \\
\end{tasks}
2、过点(1,2,3)且垂直于直线$\frac{x-1}{1}=\frac{y+1}{-1}=\frac{z-3}{3}$的平面方程为\hfill(\qquad)
\begin{tasks}(2)
    \task $1(x-1)+(y+1)+3(z-3)=0$
    \task $1(x-1)-1(y-2)+3(z-3)=0$
    \task $\frac{x-1}{1}=\frac{y+1}{2}=\frac{z-3}{3}$
    \task $\frac{x-1}{1}+\frac{y-2}{-1}+\frac{z-3}{3}=0$
    \\
\end{tasks}
2、过点(1,2,3)且垂直于平面$x-y+3z-11=0$的直线方程为\hfill(\qquad)
\begin{tasks}(2)
    \task $1(x-1)-1(y-2)+3(z-3)=0$
    \task $1(x-1)+2(y-2)+3(z-3)=0$
    \task $\frac{x-1}{1}=\frac{y-2}{-1}=\frac{z-3}{3}$
    \\
    \task $\frac{x-1}{1}+\frac{y-2}{-1}+\frac{z-3}{3}=0$
\end{tasks}
3、空间曲面$z=x^2+y^2$和$z=2-x^2-y^2$的交线在$xoy$面的投影是\hfill(\qquad)
\begin{tasks}(2)
    \task $x^2+y^2=1$
    \task $z=1$
    \task $
    \begin{cases}
        x^2+y^2=1,\\
        z=0
    \end{cases}$
    \task 以上都不正确
    \\
\end{tasks}
3、设z=$x^2y$,则$dz|_{(1,1)}=$\hfill(\qquad)
\begin{tasks}(2)
    \task $dx+dy$
    \task $2dx+dy$
    \task $2xydx+x^2dy$
    \task $x^2dx+2xydy$
    \\
\end{tasks}
4、函数$z=x^2+y^2+2x+4y-3$的驻点为\hfill(\qquad)
\begin{tasks}(2)
    \task $(0,0)$
    \task $(1,0)$
    \task $(1,-2)$
    \task $(2,-1)$
    \\
\end{tasks}
5、交换积分$\int_0^1 dx \int_x^1 f(x,y)dy$积分次序可得\hfill(\qquad)
\begin{tasks}(2)
    \task $\int_0^1dy\int_x^1f(x,y)dx$
    \task $\int_0^1dy\int_0^yf(x,y)dx$
    \task $\int_0^1dy\int_y^1f(x,y)dx$
    \task $\int_0^1dy\int_0^1f(x,y)dx$
    \\
\end{tasks}
7、在下列级数中,收敛的是\hfill(\qquad)
\begin{tasks}(2)
    \task $\sum_{n=1}^\infty ln(1+\frac{1}{n})$
    \task $\sum_{n=1}^\infty \frac{1}{n^2}$
    \task $\sum_{n=1}^\infty \frac{1}{n}$
    \task $\sum_{n=1}^\infty (\frac{1}{n^2}-\frac{1}{n})$
    \\
\end{tasks}
8、函数$f(x)=e^{2x}$展成$x$的幂级数为\hfill(\qquad)
\begin{tasks}(4)
    \task $2\sum_{n=0}^\infty \frac{x^n}{n!}$
    \task $\sum_{n=1}^\infty \frac{x^n}{n!}$
    \task $\sum_{n=0}^\infty \frac{2^n}{n!}x^n$
    \task $e^2\sum_{n=0}^\infty \frac{x^n}{n!}$
\end{tasks}
\clearpage

\end{document}
