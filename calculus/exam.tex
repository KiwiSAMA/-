\documentclass[12pt,a3paper]{article}
\usepackage[space]{ctex}
\usepackage{geometry}
\usepackage{mathtools}
\usepackage{tasks}
\settasks{
    counter-format={tsk[A].},
	label-offset={0.4em},
	label-align=left,
	column-sep={2pt},
    item-indent={1pt},
    before-skip={-0.7em},
    after-skip={-0.7em}
    }

\setlength{\parindent}{0pt}
\geometry{scale=0.8}
\title{第二学期期末数学试题(样)}
\date{}
\begin{document}
\maketitle


一、选择题
\\\\
1、点(2,-1,3)关于坐标面$xoy$的对称点是\hfill(\qquad)
\begin{tasks}(4)
    \task $(2,1,3)$
    \task $(-2,-1,3)$
    \task $(2,-1,-3)$
    \task $(-2,1,-3)$
    \\
\end{tasks}
1、点(2,-1,3)关于平面y=1的对称点是\hfill(\qquad)
\begin{tasks}(4)
    \task $(2,3,3)$
    \task $(-2,-1,3)$
    \task $(2,-1,-3)$
    \task $(-2,1,-3)$
    \\
\end{tasks}
2、过点(1,2,3)且垂直于直线$\frac{x-1}{1}=\frac{y+1}{-1}=\frac{z-3}{3}$的平面方程为\hfill(\qquad)
\begin{tasks}(2)
    \task $1(x-1)+(y+1)+3(z-3)=0$
    \task $1(x-1)-1(y-2)+3(z-3)=0$
    \task $\frac{x-1}{1}=\frac{y+1}{2}=\frac{z-3}{3}$
    \task $\frac{x-1}{1}+\frac{y-2}{-1}+\frac{z-3}{3}=0$
    \\
\end{tasks}
2、过点(1,2,3)且垂直于平面$x-y+3z-11=0$的直线方程为\hfill(\qquad)
\begin{tasks}(2)
    \task $1(x-1)-1(y-2)+3(z-3)=0$
    \task $1(x-1)+2(y-2)+3(z-3)=0$
    \task $\frac{x-1}{1}=\frac{y-2}{-1}=\frac{z-3}{3}$
    \\
    \task $\frac{x-1}{1}+\frac{y-2}{-1}+\frac{z-3}{3}=0$
\end{tasks}
3、空间曲面$z=x^2+y^2$和$z=2-x^2-y^2$的交线在$xoy$面的投影是\hfill(\qquad)
\begin{tasks}(2)
    \task $x^2+y^2=1$
    \task $z=1$
    \task $
    \begin{cases}
        x^2+y^2=1,\\
        z=0
    \end{cases}$
    \task 以上都不正确
    \\
\end{tasks}
3、设z=$x^2y$,则$dz|_{(1,1)}=$\hfill(\qquad)
\begin{tasks}(2)
    \task $dx+dy$
    \task $2dx+dy$
    \task $2xydx+x^2dy$
    \task $x^2dx+2xydy$
    \\
\end{tasks}
4、函数$z=x^2+y^2+2x+4y-3$的驻点为\hfill(\qquad)
\begin{tasks}(2)
    \task $(0,0)$
    \task $(1,0)$
    \task $(1,-2)$
    \task $(2,-1)$
    \\
\end{tasks}
5、交换积分$\int_0^1 dx \int_x^1 f(x,y)dy$积分次序可得\hfill(\qquad)
\begin{tasks}(2)
    \task $\int_0^1dy\int_x^1f(x,y)dx$
    \task $\int_0^1dy\int_0^yf(x,y)dx$
    \task $\int_0^1dy\int_y^1f(x,y)dx$
    \task $\int_0^1dy\int_0^1f(x,y)dx$
    \\
\end{tasks}
7、在下列级数中,收敛的是\hfill(\qquad)
\begin{tasks}(2)
    \task $\sum\limits_{n=1}^\infty ln(1+\frac{1}{n})$
    \task $\sum\limits_{n=1}^\infty \frac{1}{n^2}$
    \task $\sum\limits_{n=1}^\infty \frac{1}{n}$
    \task $\sum\limits_{n=1}^\infty (\frac{1}{n^2}-\frac{1}{n})$
    \\
\end{tasks}
8、函数$f(x)=e^{2x}$展成$x$的幂级数为\hfill(\qquad)
\begin{tasks}(4)
    \task $2\sum\limits_{n=0}^\infty \frac{x^n}{n!}$
    \task $\sum\limits_{n=1}^\infty \frac{x^n}{n!}$
    \task $\sum\limits_{n=0}^\infty \frac{2^n}{n!}x^n$
    \task $e^2\sum\limits_{n=0}^\infty \frac{x^n}{n!}$
\end{tasks}
\clearpage
8、若幂级数$\sum\limits_{n=0}^\infty a_n(x-2)^n$在$x=-3$处收敛,则它在$x=4$处一定是\hfill(\qquad)
\begin{tasks}(2)
    \task 条件收敛
    \task 绝对收敛
    \task 一定发散
    \task 不能确定
    \\
\end{tasks}
二、填空题
\\\\
9、设三个点的坐标为$A(1,1,2),B(2,2,1),C(3,-2,1)$,则$\overrightarrow{AB}\cdot\overrightarrow{AC}=\underline{\makebox[6em]{}} $;
\\\\
10、点$(1,0,-1)$到平面$3x+4y+5z=10$的距离为$\underline{\makebox[6em]{}}$;
\\\\
11、若$z=f(x,y)=x^2y+(y-1)\sin\frac{x}{y},则f_x(1,1)=\underline{\makebox[6em]{}}$;
\\\\
11、若$z=f(x,y)=x^3y+x^2-\ln(y+1)+3$, 则$\frac{\partial^2 z}{\partial x^2}=\underline{\makebox[6em]{}}$;
\\\\
12、若$z=z(x,y)$是由方程$2\sin(x+2y-3z)=x+2y-3z$所确定的隐函数,则$\frac{\partial z}{\partial x} + \frac{\partial z}{\partial y}=\underline{\makebox[6em]{}}$;
\\\\
13、若$f(x,y)$二阶偏导数连续,且$df(x,y)=x^2ydx+(y^3+ax^3)dy$,则$\underline{\makebox[6em]{}}$;
\\\\
14、将二次积分$\int_0^1dx\int_0^{\sqrt{1-x^2}}f(x^2+y^2)dy$化为极坐标的形式$=\underline{\makebox[6em]{}}$;
\\\\
15、若积分区域
$D = \left.
\begin{cases}
  -1 \leq x \leq 1 \\
  0 \leq y \leq \sqrt{1-x^2}
\end{cases}
\right\}
$,则$\iint\limits_D (1+xy)dxdy=\underline{\makebox[6em]{}}$;
\\\\
16、$\sum\limits_{n=1}^\infty(\frac{1}{(n+1)^2}-\frac{1}{n+2}^2)$,通过前$n$项和的极限求级数的和$S=\underline{\makebox[6em]{}}$;
\\\\
三、计算题
\\\\
17、已知$\vec{a}=\vec{i}-\vec{j}+\vec{k}$,$\vec{b}=2\vec{i}+\vec{j}+\vec{k}$,试求\\
(1)$2\vec{a}-\vec{b}$;(2)$\vec{a}\cdot\vec{b}$;(3)$\vec{a}x\vec{b}$;
\\\\
18、设$z=f(xy,x+y)$,其中$f$可微,求$\frac{\partial z}{\partial x}, \frac{\partial z}{\partial y}$.
\\\\
19、求函数$f(x,y)=y^3-x^2+6x-12y+5$的极值.
\\\\
20、计算二重积分$\iint\limits_D x^2ydxdy$,其中$D$是$y=x$,$y=2x$与$x=1$所围成的有界闭区域.
\\\\
21、计算二重积分$\iint\limits_{0\leq x\leq 1, \\0\leq y\leq 1}|x+y-1|dxdy$.
\\\\
22、用比值判别法判断级数$\sum\limits_{n=1}^\infty\frac{n}{2^n}$的敛散性.
\\\\
四、计算题
\\\\
23、计算由曲面$z=4-x^2-y^2$与$x^2+y^2=z$所围成的立方体的体积.
\\\\\\
24、用$\frac{1}{1-x}=\sum\limits_{n=0}^\infty x^n x\in (-1,1)$ 将函数$f(x)=\frac{1}{3-x}$展成(x-1)的幂的形式(写出收敛半径与收敛区间).
\end{document}
