\documentclass[11pt,a4paper]{article}
\usepackage[space]{ctex}
\usepackage{geometry}

\usepackage{tasks}
\settasks{
    counter-format={tsk[A].},
	label-offset={0.4em},
	label-align=left,
	column-sep={2pt},
    item-indent={1pt},
    before-skip={-0.7em},
    after-skip={-0.7em}
    }

\setlength{\parindent}{0pt}
\geometry{scale=0.8}
\title{第二学期期末数学试题(样)}
\date{}
\begin{document}
\maketitle



1、点(2,-1,3)关于坐标面xoy的对称点是\hfill(\qquad)
\begin{tasks}(4)
    \task $(2,1,3)$
    \task $(-2,-1,3)$
    \task $(2,-1,-3)$
    \task $(-2,1,-3)$
    \newline
\end{tasks}
1、点(2,-1,3)关于平面y=1的对称点是\hfill(\qquad)
\begin{tasks}(4)
    \task $(2,3,3)$
    \task $(-2,-1,3)$
    \task $(2,-1,-3)$
    \task $(-2,1,-3)$
    \newline
\end{tasks}
2、过点(1,2,3)且垂直于直线$\frac{x-1}{1}=\frac{y+1}{-1}=\frac{z-3}{3}$的平面方程为\hfill(\qquad)
\begin{tasks}(2)
    \task $1(x-1)+(y+1)+3(z-3)=0$
    \task $1(x-1)-1(y-2)+3(z-3)=0$
    \task $\frac{x-1}{1}=\frac{y+1}{2}=\frac{z-3}{3}$
    \task $\frac{x-1}{1}+\frac{y-2}{-1}+\frac{z-3}{3}=0$
    \newline
\end{tasks}
\end{document}
